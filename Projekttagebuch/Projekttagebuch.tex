\documentclass[accentcolor=tud0b,12pt,paper=a4]{tudreport}

\usepackage[utf8]{inputenc}
\usepackage{ngerman}
\usepackage{parcolumns}

\newcommand{\titlerow}[2]{
	\begin{parcolumns}[colwidths={1=.15\linewidth}]{2}
		\colchunk[1]{#1:} 
		\colchunk[2]{#2}
	\end{parcolumns}
	\vspace{0.2cm}
}

\title{Open Diabetes UAM Heuristik Algorithm}
\subtitle{Projekttagebuch UAM}
\subsubtitle{%
	\titlerow{Gruppe 11}{%
		Aino Schwarte <aino.schwarte@stud.tu-darmstadt.de>\\
		Anna Mees <anna.mees@stud.tu-darmstadt.de>\\
		Jan Paul Petto <janpaul.petto@stud.tu-darmstadt.de>\\
		Paul Wolfart <paul.wolfart@stud.tu-darmstadt.de>\\
		Tom Großmann <tom.grossmann@stud.tu-darmstadt.de>}
	\titlerow{Teamleiter}{Benedikt Schneider <schneider-benedikt@gmx.net>}
	\titlerow{Auftraggeber}{%
		M.Sc. Jens Heuschkel <heuschkel@tk.tu-darmstadt.de>\\
		Telecooperation\\
		Smart Urban Networks}
	\titlerow{Abgabedatum}{31.03.2019}
\institution{Bachelor-Praktikum WS 2018/2019\\Fachbereich Informatik}}

\begin{document}

	\maketitle
		
	\newpage
	\chapter*{Ereignisse}
	
	
	\begin{itemize}
	
	\item Anforderung am Anfang nicht eindeutig. Zitat aus den Folien der Themenvorstellungen: \"Die Algorithmen sowie eine Visualisierung der Ausgabe werden vorgegeben\" stattdessen müssen wir die Algorithmen selber entwickeln. 
	\item 14.01.2019 Es wird uns mitgeteilt, dass wir ein Pflichtenheft abgeben sollen.
	\item 30.01.2019 Unser Auftraggeber eröffnet eigenen Branch im Projekt.
	\item 01.02.2019 Unser Auftraggeber commitet ohne Vorwarnung eine eigene Api und macht damit den build kaputt, ohne sich um die dadurch entstandenen Fehler zu kümmern
	\item 01.02.2019 erstmalige eindeutige Anforderungen vom Auftraggeber, mit bisher nicht erwähnten Aufgaben.
	
	
	
	\end{itemize}		

	
	
	
\end{document}
