\documentclass[accentcolor=tud0b,12pt,paper=a4]{tudreport}

\usepackage[utf8]{inputenc}
\usepackage{ngerman}
\usepackage{parcolumns}

\newcommand{\titlerow}[2]{
	\begin{parcolumns}[colwidths={1=.15\linewidth}]{2}
		\colchunk[1]{#1:} 
		\colchunk[2]{#2}
	\end{parcolumns}
	\vspace{0.2cm}
}

\title{Open Diabetes UAM Heuristik Algorithm}
\subtitle{Projekttagebuch UAM}
\subsubtitle{%
	\titlerow{Gruppe 11}{%
		Aino Schwarte <aino.schwarte@stud.tu-darmstadt.de>\\
		Anna Mees <anna.mees@stud.tu-darmstadt.de>\\
		Jan Paul Petto <janpaul.petto@stud.tu-darmstadt.de>\\
		Paul Wolfart <paul.wolfart@stud.tu-darmstadt.de>\\
		Tom Großmann <tom.grossmann@stud.tu-darmstadt.de>}
	\titlerow{Teamleiter}{Benedikt Schneider <schneider-benedikt@gmx.net>}
	\titlerow{Auftraggeber}{%
		M.Sc. Jens Heuschkel <heuschkel@tk.tu-darmstadt.de>\\
		Telecooperation\\
		Smart Urban Networks}
	\titlerow{Abgabedatum}{31.03.2019}
\institution{Bachelor-Praktikum WS 2018/2019\\Fachbereich Informatik}}

\begin{document}

	\maketitle
		
	\newpage
	\chapter*{Einträge}
	
	
	\begin{itemize}
	
\item Zu Beginn des Bachelorpraktikums war die Aufgabenstellung irreführend dargestellt. Sowohl die Folien der Themenvorstellung, als auch die ersten Treffen mit dem Auftraggeber, beinhalteten noch, “Die Algorithmen sowie eine Visualisierung der Ausgabe werden vorgegeben”. Unser Auftraggeber teilte uns erst später mit, dass es unsere Aufgabe sei die Entwicklung der Algorithmen und deren Visualisierung komplett zu übernehmen.

\item Zur Einarbeitung und Entwicklung der Algorithmen und gewünschter Features mussten wir uns durch undokumentierten und teilweise unkommentierten Code anderer Programmiersprachen durcharbeiten.

\item Am 14.01.2019 wird uns mitgeteilt, dass wir ein Pflichtenheft abgeben sollen, da User Stories nur für die Visualisierung in Nightscout und für das Plotten der Algorithmen sinnvoll sind.

\item Am 30.01.2019 eröffnet unser Auftraggeber einen eigenen Branch im Projekt.

\item Am 01.02.2019 comittet unser Auftraggeber ohne Vorwarnung eine eigene Api. Dies hatte zur Folge, dass ein Build des Projektes nicht möglich war. Eine Nachricht des Auftraggebers bezüglich der entstandenen Fehler erfolgte nicht.

\item Am 01.02.2019 gab der Auftraggeber erstmals eine Liste der Anforderungen der Software bekannt, die einige Features beinhielt, die bisher nicht erwähnt wurden.


	
	\end{itemize}		

	
	
	
\end{document}
